\chapter{Un guide}

\section{Introduction}
Chaque chapitre doit commencer par une courte introduction et une courte conclusion. Suivez soigneusement les conseils de votre superviseur lorsque vous rédigez vos introductions et vos conclusions. 

\section{La structure générale}
Un rapport comprend une introduction générale, suivi d'un chapitre de l’état de l'art. Dans le troisième chapitre, vous expliquez l'architecture ou la méthodologie que vous avez utilisée. La mise en œuvre (l’implémentation) est expliquée et les résultats sont discutés dans le chapitre 4. Dans la conclusion générale [chapitre 5], décrivez la contribution de votre projet, ainsi que les critiques et les limites de votre travail, suivies d'éventuelles extensions et perspectives.

Ajoutez toutes les références utilisées à la fin de votre rapport après la conclusion générale. Enfin, vous pouvez ajouter vos annexes (si vous en avez) après les références.



\section{Du pronom désignant l'auteur du rapport}
Utilisez « nous » pour désigner l'auteur du mémoire.

\textbf{Exemple:} Dans ce chapitre, nous introduisons la notation utilisée pour le reste du mémoire.

\section{Du pronom désignant le lecteur ou une personne en général}
Utilisez « on » pour désigner le lecteur ou une personne en général.

\textbf{Exemple 1:} \textit{On} note que cette liste est longue.

\textbf{Exemple 2:} Dans la phase de programmation, \textit{on} doit tout d'abord obtenir une spécification précise du programme.

\section{Conclusion}
Ne dépassez pas 5 phrases dans les conclusions de vos chapitres. 