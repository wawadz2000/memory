\documentclass[12pt]{book}
\usepackage{graphicx}
\usepackage{tabularx}
\usepackage{authblk}
\usepackage{color}
\usepackage{pdfpages}
\usepackage{hyperref}
\usepackage{lipsum}
\usepackage[Lenny]{fncychap}
\usepackage[french]{babel}
\usepackage{listings}
\definecolor{dkgreen}{rgb}{0,0.6,0}
\definecolor{gray}{rgb}{0.5,0.5,0.5}
\definecolor{mauve}{rgb}{0.58,0,0.82}

\lstset{frame=tb,
  language=Python,
  aboveskip=3mm,
  belowskip=3mm,
  showstringspaces=false,
  columns=flexible,
  basicstyle={\small\ttfamily},
  numbers=none,
  numberstyle=\tiny\color{gray},
  keywordstyle=\color{blue},
  commentstyle=\color{dkgreen},
  stringstyle=\color{mauve},
  breaklines=true,
  breakatwhitespace=true,
  tabsize=3
}
\ChNameUpperCase
\ChNumVar{\fontsize{40}{42}\usefont{OT1}{ptm}{m}{n}\selectfont}
\ChTitleVar{\Large\sc}
\begin{document}
\includepdf[pages=-]{Page de garde-1-4.pdf}
\tableofcontents
\listoffigures
\newpage
\section{Introduction Générale}
\chapter{Apprentissage automatique et deep learning}
\chapter{Conception et Modélisation UML}
\chapter{Implémentation}
\chapter{Conclusion Générale}
\begin{thebibliography}{9}
    \bibitem{chatgpt}
    \href{https://chat.openai.com}{chat.openai.com}
    
    \bibitem{book}
    Chantal Morley, Jean Hugues, Bernard Leblanc (2006) \emph{UML2 pour l’analyse d’un système d’information}, Dunod.
\end{thebibliography}
    
\end{document}
