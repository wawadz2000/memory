\chapter{Introduction Générale}
Chaque rapport doit commencer par une introduction générale dans laquelle le contexte du projet est clairement expliqué. Cette introduction devrait également inclure l'objectif du projet et le plan du reste du rapport. Cette introduction ne devrait pas dépasser 2 pages. Soyez concis et clair, et écrivez uniquement ce qui est nécessaire à écrire.

\section{Introduction }
L'évaluation médicale est un processus fondamental dans la pratique clinique, visant à évaluer l'état de santé d'un patient et à déterminer les meilleures stratégies de prise en charge.
 Dans ce contexte, l'Évaluation Médicale Compréhensive (EMC) émerge comme une approche intégrative, cherchant à appréhender la globalité des besoins du patient au-delà de la simple symptomatologie.

\section{Diagnostic médical }
Le diagnostic médical, fruit de la science et de l'expertise, il éclaire le chemin vers la compréhension des symptôme, guidant ainsi le parcours vers la guérison .
\newpage
\subsection{Historique diagnostic }
A ses origines, la médecine était essentiellement magico-religieuse, le diagnostic appartenait aux devins, aux oracles et aux prêtres. Avec Hippocrate, la conception de la médecine devint plus rationnelle et le diagnostic (encore volontiers confondu avec le pronostic) fondé sur un examen clinique minutieux.
Au XVII et au XVIII siècle, la démonstration de la circulation du sang par Harvey introduisit la médecine dans le monde de la mécanique ; Morgagni fonda l’anatomopathologie et la méthode anatomoclinique prit naissance. Son essor se fit dans le cadre du développement de la sémiologie, lui-même lié à la découverte de la percussion à la fin du XVIIIe siècle (Auenbrugger) et à celle de l’auscultation (Laennec) au début du XIX siècle.
Au cours de ce même siècle elle atteignit sa plénitude avec l’apparition de l’anatomopathologie microscopique (Broca). 
A la méthode anatomoclinique vint s’adjoindre avec Claude
Bernard la méthode physio clinique introduisant la physiopathologie et la biologie. Le XIXe siècle vit les débuts de la bactériologie, de l’immunologie et de l’imagerie médicale.
Le XXe siècle fut marqué par le développement de l’endoscopie et des prélèvements biopsiques rendus nécessaires aussi par l’essor de la chirurgie, l’artériographie précédant les méthodes d’explorations non traumatisantes : explorations isotopiques, tomodensitométrie et résonance magnétique nucléaire, l’usage de l’échographie et de l’effet doppler.

La biochimie, la génétique permirent progressivement de substituer à l’étalon or de la confrontation anatomoclinique des critères plus subtils, le passage de l’échelon microscopique au stade de l’échelon moléculaire modifiant progressivement mais radicalement nos conceptions du diagnostic en médecine. Ainsi la médecine des deux derniers siècles a vu se confirmer la rationalité de la démarche médicale puis son développement scientifique, des lésions tissulaires on est passé à la biologie moléculaire.
\subsection{Définitions diagnostic}
Le diagnostic est le processus d'évaluation d'un état de fonctionnement donné. Si cet état est comparé avec un état de référence, il s'agit d'évaluation de dérive de fonctionnement. 
\subsection{Les étapes de diagnostic}

\begin{itemize}
\item \textbf{Collecte d'informations} :
Recueillir des données pertinentes liées au problème ou à la situation, que ce soit des symptômes, des données techniques, des antécédents, etc.

\item \textbf{Identification du problème} :
Analyser les informations collectées pour déterminer la nature du problème. Cela peut impliquer la comparaison des données avec des normes ou des critères établis.

\item \textbf{Élaboration d'hypothèses} :
Formuler des hypothèses sur les causes possibles du problème en se basant sur les informations disponibles.

\item \textbf{Tests et investigations} :
Mettre en place des tests ou des investigations pour valider ou invalider les hypothèses formulées. Cela peut inclure des examens médicaux, des tests techniques, des simulations, etc.

\item \textbf{Analyse des résultats} :
Examiner les résultats des tests et des investigations afin de confirmer la cause du problème ou de revoir les hypothèses si nécessaire.

\item \textbf{Établissement du diagnostic} :
Formuler un diagnostic final en identifiant la cause principale du problème ou de la situation, en tenant compte de toutes les informations et des résultats obtenus.

\item \textbf{Proposition de solutions} :
Proposer des solutions ou des recommandations pour résoudre le problème diagnostiqué. Cela peut impliquer des traitements médicaux, des ajustements techniques, des interventions psychologiques, etc.

\item \textbf{Suivi et évaluation} :
Mettre en place un suivi pour évaluer l'efficacité des solutions proposées et ajuster si nécessaire. Cela peut également inclure des mesures préventives pour éviter la récurrence du problème .

\end{itemize}

\section{Symptômes médicaux  }
Les symptômes médicaux sont souvent les signaux précurseurs d'un état de santé sous-jacent, nécessitant une évaluation médicale approfondie pour établir un diagnostic précis et élaborer un plan de traitement adapté.
\subsection{Définitions les symptômes }
Les symptômes se réfèrent à des signes ou manifestations d'une maladie, d'un trouble ou d'une condition médicale. Ce sont des indicateurs observables ou ressentis par le patient, ainsi que détectés par les professionnels de la santé. Les symptômes médicaux peuvent varier en fonction de la maladie ou du trouble et Il est important de noter que les symptômes ne sont pas la maladie elle-même, mais plutôt des signaux qui indiquent la présence possible d'un problème de santé. Ils peuvent être utilisés pour aider les professionnels de la santé à poser un diagnostic et à recommander un plan de traitement approprié.
\subsection{Exemples de symptômes }
\begin{itemize}
\item La fatigue est définie comme une sensation d'épuisement survenant durant ou après une activité habituelle. Une cause organique ou psychiatrique est retrouvée dans la majorité des cas.

\item Le vomissement est le rejet du contenu de l'estomac par la bouche. Il correspond à un réflexe mécanique de défense de l'organisme destiné à vider l'estomac. Il est possible de vomir des aliments, de la bile ou beaucoup plus rarement, du sang.

\item La fièvre est une température corporelle anormalement élevée, qui dépasse 38°C. 

\item La toux est l'expiration brusque et sonore de l'air contenu dans les poumons provoquée par une irritation des voies respiratoires.

\item La douleur thoracique désigne toute douleur ou toute sensation anormale et pénible localisée dans la zone du thorax.

\item La perte d'appétit est la perte de l'envie de manger, peut aussi être appelée anorexie.

\item Les céphalées sont un problème très courant. Il existe plusieurs types de céphalées, les céphalées de tension étant les plus fréquentes. Bien qu’en général bénignes, les céphalées peuvent être le symptôme d’une maladie grave.
\end{itemize}

\section{Pronostic médical }
Le pronostic médical, étroitement lié aux symptômes médicaux présentés, offre une perspective éclairante sur l'évolution potentielle de la condition de santé, permettant aux professionnels de la santé de prendre des décisions informées pour un plan de traitement optimal.
\subsection{Définition Pronostic médical  }
Le pronostic médical est une évaluation anticipée de l'évolution probable d'une maladie chez un patient, basée sur des données cliniques et des connaissances médicales, afin de prédire les perspectives de guérison, de rémission, de stabilisation ou de progression de la condition.
\subsection{Les type Pronostic médical }
\begin{itemize}
\item \textbf{Pronostic vital} : Il concerne la probabilité de survie d'un patient, souvent exprimée en termes de taux de survie à un certain nombre d'années.

\item \textbf{Pronostic qualité de vie} : Il prend en compte l'impact de la maladie ou du traitement sur la qualité de vie globale du patient.

\item \textbf{Pronostic génétique} : Il se base sur des facteurs génétiques et évalue la probabilité de développement de maladies héréditaires.

\item \textbf{Pronostic fonctionnel} : Il évalue la capacité du patient à maintenir une fonction normale ou à retrouver une fonction normale après un traitement.
\end{itemize}

\section{La relation entre diagnostic, les symptômes et Pronostic médical }
La relation entre le diagnostic, les symptômes et le pronostic médical est cruciale, les symptômes fournissant des indices précieux pour orienter le processus diagnostique, tandis que le diagnostic éclaire le pronostic en permettant une évaluation anticipée de l'évolution probable de la maladie, guider ainsi les choix de traitement et les décisions médicales.
\section{Conclusion}
Cette étude souligne l'interdépendance cruciale entre le diagnostic, les symptômes et le pronostic médical dans le contexte de la médecine moderne. Une compréhension approfondie de ces éléments est essentielle pour assurer des soins de qualité et des résultats positifs pour les patients. 
Le chapitre suivant présente l’apprentissage automatique afin de pouvoir analyser les données médicales.
